%%%%%%%%%%%%%%%%%%%%%%%%%%%%%%%%%%%%%%%%%%%%%%%%%%%
%% P3: Phenomenology of Particle Physics                         
%%
%% Author:  André Rubbia                   		 
%%
%% Figure 2.11 Sketch of the scattering of a uniform beam of particles with flux per unit area $\varphi$ impinging on a thin target of thickness $\d x$ and total area $A$.
%%
%% This work is licensed under the Creative Commons Attribution 4.0 International License. 
%% To view a copy of this license, visit http://creativecommons.org/licenses/by/4.0/ or 
%% send a letter to Creative Commons, PO Box 1866, Mountain View, CA 94042, USA.
%%
%%%%%%%%%%%%%%%%%%%%%%%%%%%%%%%%%%%%%%%%%%%%%%%%%%%

\documentclass[a4paper,10pt]{article}
\usepackage{hyperref}
\usepackage{hyperxmp}
\usepackage[
    type={CC},
    version={4.0}
]{doclicense}

\usepackage[T1]{fontenc}
\usepackage[utf8]{inputenc}
\usepackage{lmodern}
\usepackage[labelfont=bf]{caption}
\usepackage{upgreek}
\usepackage{amssymb}
\usepackage{amsmath}

\usepackage{tikz}
\usepackage{pgfplots}
\pgfplotsset{compat=1.17}
\usepgfplotslibrary{ternary}
\usepgfplotslibrary{fillbetween}
\usepgfplotslibrary{external}

\usepackage{pst-3dplot}
\usepackage{auto-pst-pdf}

\def\d{\mathrm{d}}

\begin{document}

%%%%%%%%%%%%%%%   FIGURE  %%%%%%%%%%%%%%%%%%%%%%%%%%%%%%
\begin{figure}[htb]
\centering\mbox{
\psset{unit=0.65}
\begin{pspicture}(-4,-4)(4,4)%
\psset{Alpha=30,Beta=20}
%%%\pstThreeDCoor[xMax=2,yMax=2,zMax=2]
{\psset{arrows=->,linewidth=2pt,fillstyle=none}
\pstThreeDLine[linecolor=magenta, linewidth=2pt](0.5,4,2.5)(-6,3,5.5)
\pstThreeDLine[linecolor=magenta, linewidth=2pt](0.5,10,2.5)(-3,10,2.5)
}

\psBox[showInside=false](0,0,0){1}{7}{12}

{\psset{arrows=->,linewidth=2pt,fillstyle=none}
\pstThreeDLine[linecolor=magenta, linewidth=2pt](6,0,2.5)(3,0,2.5)
\pstThreeDLine[linecolor=magenta, linewidth=2pt](6,1,2.5)(3,1,2.5)
\pstThreeDLine[linecolor=magenta, linewidth=2pt](6,2,2.5)(3,2,2.5)
\pstThreeDLine[linecolor=magenta, linewidth=2pt](6,3,2.5)(3,3,2.5)
%
\pstThreeDLine[linecolor=magenta, linewidth=2pt](6,4,2.5)(0.5,4,2.5)
%
\pstThreeDLine[linecolor=magenta, linewidth=2pt](6,5,2.5)(3,5,2.5)
\pstThreeDLine[linecolor=magenta, linewidth=2pt](6,6,2.5)(3,6,2.5)
\pstThreeDLine[linecolor=magenta, linewidth=2pt](6,7,2.5)(3,7,2.5)
\pstThreeDLine[linecolor=magenta, linewidth=2pt](6,8,2.5)(3,8,2.5)
\pstThreeDLine[linecolor=magenta, linewidth=2pt](6,9,2.5)(3,9,2.5)
\pstThreeDLine[linecolor=magenta, linewidth=2pt](6,10,2.5)(0.5,10,2.5)
\pstThreeDLine[linecolor=magenta, linewidth=2pt](6,11,2.5)(3,11,2.5)
\pstThreeDLine[linecolor=magenta, linewidth=2pt](6,12,2.5)(3,12,2.5)
}

{\psset{arrows=<->,linewidth=1pt,fillstyle=none}
\pstThreeDLine[linecolor=black, linewidth=1pt](0,12.5,0)(1,12.5,0)
}

{\psset{linewidth=2pt, linestyle=dashed}
\pstThreeDLine[linecolor=magenta](0.5,4,2.5)(-4.4,2.15,3.9)
\pstThreeDLine[linecolor=magenta](0.5,10,2.5)(-1.25,10,2.5)
}

\pstThreeDDot(0.5,1,5.5)
\pstThreeDDot(0.5,1,2.5)
\pstThreeDDot(0.5,0.5,0.5)
\pstThreeDDot(0.5,2,3.5)
\pstThreeDDot(0.5,4,4.5)
\pstThreeDDot(0.5,7.5,2)
\pstThreeDDot(0.5,9,0.5)
\pstThreeDDot(0.5,3,2)
\pstThreeDDot(0.5,4,1.5)
\pstThreeDDot(0.5,7.5,3)
\pstThreeDDot(0.5,1,6)
\pstThreeDDot(0.5,7,5.5)
\pstThreeDDot(0.5,5.5,3)
\pstThreeDDot(0.5,9.5,4.5)
\pstThreeDDot(0.5,7.5,2.5)
\pstPlanePut[plane=yz](0,2.8,2.4){\Large $\sigma$}
\pstPlanePut[plane=zx](0.5,13,-0.5){$\d x$}
\pstPlanePut[plane=zx](1,4,6){Area $A$}
\pstPlanePut[plane=zx](8,6,2.5){Flux $\varphi$}

\end{pspicture}
}
\caption{Sketch of the scattering of a uniform beam of particles with flux per unit area $\varphi$
impinging on a thin target of thickness $\d x$ and total area $A$. The scattering centers are shown as
black dots in the target. The cross-section of each
scatterer is labeled $\sigma$.}
\end{figure}
%%%%%%%%%%%%%%%  END FIGURE  %%%%%%%%%%%%%%%%%%%%%%%%%%%%%%%

\vskip 5cm
\doclicenseThis

\end{document}
