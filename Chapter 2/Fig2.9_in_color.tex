%%%%%%%%%%%%%%%%%%%%%%%%%%%%%%%%%%%%%%%%%%%%%%%%%%%
%% P3: Phenomenology of Particle Physics                         
%%
%% Author:  André Rubbia                   		 
%%
%% Figure 2.9 An incident particle that crosses the surface $\d\sigma$ will be scattered in the solid angle $\d \Omega$.
%%
%% This work is licensed under the Creative Commons Attribution 4.0 International License. 
%% To view a copy of this license, visit http://creativecommons.org/licenses/by/4.0/ or 
%% send a letter to Creative Commons, PO Box 1866, Mountain View, CA 94042, USA.
%%
%%%%%%%%%%%%%%%%%%%%%%%%%%%%%%%%%%%%%%%%%%%%%%%%%%%

\documentclass[a4paper,10pt]{article}
\usepackage{hyperref}
\usepackage{hyperxmp}
\usepackage[
    type={CC},
    version={4.0}
]{doclicense}

\usepackage[T1]{fontenc}
\usepackage[utf8]{inputenc}
\usepackage{lmodern}
\usepackage[labelfont=bf]{caption}
\usepackage{upgreek}
\usepackage{amssymb}
\usepackage{amsmath}

\usepackage{tikz}
\usepackage{pgfplots}
\pgfplotsset{compat=1.17}
\usepgfplotslibrary{ternary}
\usepgfplotslibrary{fillbetween}
\usepgfplotslibrary{external}

\usepackage{pst-3dplot}
\usepackage{auto-pst-pdf}

\def\d{\mathrm{d}}

\begin{document}

%%%%%%%%%%%%%%%   FIGURE  %%%%%%%%%%%%%%%%%%%%%%%%%%%%%%
\begin{figure}[htb]
\begin{center}
\centering\mbox{
\psset{unit=0.75}
\begin{pspicture}(-4,-4)(4,4)%
\psset{Alpha=-20,Beta=25}
%\psset{Alpha=0,Beta=90}
\pstThreeDSphere[increment=18,SegmentColor={[cmyk]{0,0,0,0}}](-0.1,0,0){4}
\pstThreeDLine[linecolor=black, linewidth=1pt](-7,0,0)(15,0,0)
\pstThreeDDot(-0.1,0,0)
\pstThreeDDot(10,0,0)

% dsigma
\pscustom[fillstyle=solid,fillcolor=blue]{
\pstThreeDCircle[beginAngle=70,endAngle=110](10,0,0)(0,1,0)(0,0,1)
\pstThreeDLine(10,-0.342020,0.939693)(10,-0.68,1.8)
\pstThreeDCircle[beginAngle=110,endAngle=70](10,0,0)(0,2,0)(0,0,2)
\pstThreeDLine(10,0.68,1.8)(10,0.342020,0.939693)
}

% domega
\pstThreeDCircle(-2.7,0,0)(0,3,0)(0,0,3)
\pstThreeDCircle(-1.7,0,0)(0,3.7,0)(0,0,3.7)

%\pscustom[fillstyle=solid,fillcolor=gray]{
%\pstThreeDCircle[beginAngle=-70,endAngle=120](-2.7,0,0)(0,3,0)(0,0,3)
%\pstThreeDLine(-2.7,-1.02,2.82)(-1.7,-1.26,3.47)
%\pstThreeDCircle[beginAngle=-70,endAngle=120](-1.7,0,0)(0,3.7,0)(0,0,3.7)
%\pstThreeDLine(-1.7,1.26,3.47)(-2.7,1.02,2.82)
%}

\pscustom[fillstyle=solid,fillcolor=blue]{
\pstThreeDCircle[beginAngle=70,endAngle=110](-2.7,0,0)(0,3,0)(0,0,3)
\pstThreeDLine(-2.7,-1.02,2.82)(-1.7,-1.26,3.47)
\pstThreeDCircle[beginAngle=110,endAngle=70](-1.7,0,0)(0,3.7,0)(0,0,3.7)
\pstThreeDLine(-1.7,1.26,3.47)(-2.7,1.02,2.82)
}
\pstThreeDLine[linecolor=blue](-0.1,0,0)(-2.7,-1.02,2.82)
\pstThreeDLine[linecolor=blue](-0.1,0,0)(-1.7,-1.26,3.47)
\pstThreeDLine[linecolor=blue](-0.1,0,0)(-2.7,1.02,2.82)
\pstThreeDLine[linecolor=blue](-0.1,0,0)(-1.7,1.26,3.47)

% trajectory
\pstThreeDLine[linewidth=2pt,arrowscale=1.5,linecolor=magenta,linearc=6]{->}(15,0,1.5)(0,0,1.5)(-6.25,0,6)


%\pstThreeDCircle[linewidth=3pt,linecolor=blue, beginAngle=90,endAngle=60](8,0,0)(0,1.5,0)(0,0,1.5)

%% dsigma
\pstThreeDSquare(10,-3,-3)(0,6,0)(0,0,6)
\pstThreeDCircle(10,0,0)(0,1,0)(0,0,1)
\pstThreeDCircle(10,0,0)(0,2,0)(0,0,2)

% b
{\psset{arrows=<->,linewidth=1pt,fillstyle=none, linecolor=red}
\pstThreeDLine(14,0,0)(14,0,1.5)
}

% phi angle
{\psset{arrowscale=1.5,arrows=->}
\psset{linecolor=red,linewidth=1.5pt,beginAngle=0,endAngle=270}
\pstThreeDEllipse(-6,0,0)(0,0,1)(0,1,0)
}
% theta angle
{\psset{arrowscale=1.5,arrows=->}
\psset{linecolor=red,linewidth=1.5pt,beginAngle=180,endAngle=120}
\pstThreeDEllipse(-0.1,0,0)(1,0,0)(0,0,1)
}

{\psset{arrowscale=1.5,arrows=->}
\pstThreeDLine[linecolor=black](11,0,3.)(10,0,2)
\pstThreeDLine[linecolor=black](-0.5,0,5)(-1.5,0,4)
}

\end{pspicture}
\setlength{\unitlength}{1cm}
\put(-14,6.5){$\d\sigma = b\d b\d\phi$}
\put(-5.5,6.5){$\d\Omega = \d\cos\theta\d\phi$}
\put(-15.25,4.95){\color{red}$b$}
\put(-4.15,3.25){\color{red}$\theta$}
\put(-0.5,3.25){\color{red}$\phi$}
}
\caption{An incident particle that crosses the surface
$\d\sigma$ will be scattered in the solid angle $\d \Omega$.}
\end{center}
\end{figure}
%%%%%%%%%%%%%%%  END FIGURE  %%%%%%%%%%%%%%%%%%%%%%%%%%%%%%%

\vskip 5cm
\doclicenseThis

\end{document}
