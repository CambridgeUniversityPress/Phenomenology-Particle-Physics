%%%%%%%%%%%%%%%%%%%%%%%%%%%%%%%%%%%%%%%%%%%%%%%%%%%
%% P3: Phenomenology of Particle Physics                         
%%
%% Author:  André Rubbia                   		 
%%
%% Figure 4.6 Illustration of the transformation of the wave function under  the rotation $R$ of the axes .
%%
%% This work is licensed under the Creative Commons Attribution 4.0 International License. 
%% To view a copy of this license, visit http://creativecommons.org/licenses/by/4.0/ or 
%% send a letter to Creative Commons, PO Box 1866, Mountain View, CA 94042, USA.
%%
%%%%%%%%%%%%%%%%%%%%%%%%%%%%%%%%%%%%%%%%%%%%%%%%%%%

\documentclass[a4paper,10pt]{article}
\usepackage{hyperref}
\usepackage{hyperxmp}
\usepackage[
    type={CC},
    version={4.0}
]{doclicense}

\usepackage[T1]{fontenc}
\usepackage[utf8]{inputenc}
\usepackage{lmodern}
\usepackage[labelfont=bf]{caption}
\usepackage{upgreek}

\usepackage{tikz}
\usepackage{pgfplots}
\pgfplotsset{compat=1.17}
\usepgfplotslibrary{ternary}
\usepgfplotslibrary{fillbetween}
\usepgfplotslibrary{external}

\def\d{\mathrm{d}}

\begin{document}

%%%%%%%%%%%%%%%   FIGURE  %%%%%%%%%%%%%%%%%%%%%%%%%%%%%%
\begin{figure}[htb]
    \centering
    \begin{tikzpicture}
        \draw[thick,->] (0,0) -- (4,0) node[right] {$x$};
        \draw[thick,->] (0,0) -- (0,4) node[above] {$y$};a
        \draw[dashed,thick,->] (0,0) -- +(60:4) node[left] {$y'$};
        \draw[dashed,thick,->] (0,0) -- +(-30:4) node[right] {$x'$};
        \draw[dashed] (2,0) -- (2,2);
        \draw[dashed] (0,2) -- (2,2);
        \filldraw (2,2) circle (1mm) node [right] {$\Psi(\vec x)$};
%        \filldraw (2,2) circle (1mm) node [right] {$\phi(\vec x)$};
        \draw[thick,->] (6,0) -- (10,0) node[right] {$x'$};
        \draw[thick,->] (6,0) -- (6,4) node[above] {$y'$};
        \draw[dashed] (6,2.73) -- (6.73,2.73);
        \draw[dashed] (6.73,0) -- (6.73,2.73);
        \filldraw (6.73,2.73) circle (1mm) node [right] {$\Psi'(\vec x\,') =
        \Psi(R^{-1}\vec x\,')$};
%        \filldraw (6.73,2.73) circle (1mm) node [right] {$\phi'(\vec x\,') =
%        \phi(R^{-1}\vec x\,')$};
        \draw[ultra thick,red,->] (0,3) arc(90:60:3);
        \draw[ultra thick,red,->] (3,0) arc(0:-30:3);
    \end{tikzpicture}
    \caption{Illustration of the transformation of the wave function under
    the rotation $R$ of the axes (note that we have
    chosen the clockwise definition of the rotation).}
\end{figure}
%%%%%%%%%%%%%%%   END FIGURE  %%%%%%%%%%%%%%%%%%%%%%%%%%%%%%
%
\vskip 5cm
\doclicenseThis

\end{document}
