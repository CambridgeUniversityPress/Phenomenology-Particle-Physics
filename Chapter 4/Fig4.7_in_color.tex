%%%%%%%%%%%%%%%%%%%%%%%%%%%%%%%%%%%%%%%%%%%%%%%%%%%
%% P3: Phenomenology of Particle Physics                         
%%
%% Author:  André Rubbia                   		 
%%
%% Figure 4.7 Sketch of the Stern--Gerlach experiment.
%%
%% This work is licensed under the Creative Commons Attribution 4.0 International License. 
%% To view a copy of this license, visit http://creativecommons.org/licenses/by/4.0/ or 
%% send a letter to Creative Commons, PO Box 1866, Mountain View, CA 94042, USA.
%%
%%%%%%%%%%%%%%%%%%%%%%%%%%%%%%%%%%%%%%%%%%%%%%%%%%%

\documentclass[a4paper,10pt]{article}

\usepackage[T1]{fontenc}
\usepackage[utf8]{inputenc}
\usepackage{lmodern}
\usepackage[labelfont=bf]{caption}
\usepackage{upgreek}
\usepackage{amssymb}
\usepackage{amsmath}

\usepackage{tikz}
\usepackage{pgfplots}
\pgfplotsset{compat=1.17}
\usepgfplotslibrary{ternary}
\usepgfplotslibrary{fillbetween}
\usepgfplotslibrary{external}

\usepackage{pst-3dplot}
\usepackage{auto-pst-pdf}

\def\d{\mathrm{d}}

\begin{document}

%%%%%%%%%%%%%%%   FIGURE  %%%%%%%%%%%%%%%%%%%%%%%%%%%%%%
\begin{figure}[htb]
\centering
\psset{unit=0.75}
\begin{pspicture}(-5,-1)(5,20)
\psset{Alpha=30,Beta=20}
%\pstThreeDCoor[xMin=-7,xMax=1,yMin=0,yMax=1,zMin=0,zMax=1]
%%\psdot[dotscale=10](0,0)
%% \pstThreeDCircle[ O p t i o n s ] ( c x , c y, c z ) ( u x , u y, u z ) ( v x , v y, v z )
% back screen
%\psset{linecolor=gray,linewidth=1pt,fillcolor=black!10,fillstyle=solid}
\pstThreeDSquare(-7,-2.5,0)(0,5,0)(0,0,5)
% magnet
\psset{linecolor=blue,linewidth=1pt,fillstyle=none}
\psBox[showInside=false](-2,-2,0){7}{4}{2}
\psBox[showInside=false](-2,0,0){7}{2}{2}
{
\psset{linecolor=black,linewidth=1pt,fillstyle=solid}
\pstThreeDSquare(-2,1,4)(7,0,0)(0,-1,2)
\pstThreeDSquare(-2,1,4)(7,0,0)(0,1,2)
}
% trajectory
{\psset{arrows=->,linewidth=3pt,fillstyle=none}
\pstThreeDLine[linecolor=magenta, linewidth=3pt](6,0,2.5)(-6,0,2.5)
}
\psBox[showInside=false](-2,0,6){7}{2}{2}
\psBox[showInside=false](-2,2,0){7}{4}{2}
%% solenoid
%% \pstIIIDCylinder(x,y,z){radius}{height}
\psset{linecolor=black,linewidth=1pt}
% source
\pstThreeDSquare(5.7,0,2)(0,1,0)(0,0,1)
\pstThreeDSquare(6,0,2)(0,1,0)(0,0,1)
\psBox[showInside=false](6.2,0,2){1}{1}{1}
% labels
\pstPlanePut[plane=yz](-7,-1,0.5){Screen}
\pstPlanePut[plane=yz](4,-1,0.5){S}
\pstPlanePut[plane=yz](4,-1,5.5){N}
\pstPlanePut[plane=yz](8,0,2){Source}
% image
\parametricplotThreeD[xPlotpoints=200,
   linecolor=blue,%
   linewidth=1.5pt,plotstyle=curve](0,4){%
   -7 -2 t add 45 t mul sin 3 add}%degrees
\parametricplotThreeD[xPlotpoints=200,
   linecolor=blue,%
   linewidth=1.5pt,plotstyle=curve](0,4){%
   -7 -2 t add 45 t mul sin -1 mul 3 add }%degrees
\end{pspicture}
\caption{Sketch of the Stern--Gerlach experiment.}
\end{figure}
%%%%%%%%%%%%%%%  END FIGURE  %%%%%%%%%%%%%%%%%%%%%%%%%%%%%%%

\end{document}
