%%%%%%%%%%%%%%%%%%%%%%%%%%%%%%%%%%%%%%%%%%%%%%%%%%%
%% P3: Phenomenology of Particle Physics                         
%%
%% Author:  André Rubbia                   		 
%%
%% Figure 4.10 Graphical representation of the perturbation series in the $i\rightarrow f$ transition to first order, second order, and third order.
%%
%% This work is licensed under the Creative Commons Attribution 4.0 International License. 
%% To view a copy of this license, visit http://creativecommons.org/licenses/by/4.0/ or 
%% send a letter to Creative Commons, PO Box 1866, Mountain View, CA 94042, USA.
%%
%%%%%%%%%%%%%%%%%%%%%%%%%%%%%%%%%%%%%%%%%%%%%%%%%%%

\documentclass[a4paper,10pt]{article}
\usepackage{hyperref}
\usepackage{hyperxmp}
\usepackage[
    type={CC},
    version={4.0}
]{doclicense}

\usepackage[T1]{fontenc}
\usepackage[utf8]{inputenc}
\usepackage{lmodern}
\usepackage[labelfont=bf]{caption}
\usepackage{upgreek}

\usepackage{tikz}
\usepackage{pgfplots}
\pgfplotsset{compat=1.17}
\usepgfplotslibrary{ternary}
\usepgfplotslibrary{fillbetween}
\usepgfplotslibrary{external}

\def\d{\mathrm{d}}

\begin{document}

%%%%%%%%%%%%%%%%   FIGURE  %%%%%%%%%%%%%%%%%%%%%%%%%%%%%%
\begin{figure}[htb]
\begin{center}
\begin{tikzpicture}

\draw[dashed] (5,0)--(10,0);
\draw[dashed] (7,0)--+(30:4);
\draw[red,thick,->] (8.7,0) arc[radius=1.7, start angle=0, end angle=30];
\draw[blue,fill] (7,0) circle (3pt);
\node at (7,-0.6) {$V_{fi}$};
\draw[very thick,->] (4,0)--+(0:1) node[above] {$i$};
\draw[very thick,->] (9,1.15)--+(30:1) node[above] {$f$};
\node at (11,0) {$+$};
\node at (7,-5) {$+$};

\begin{scope}[shift={(8,0)}]
\draw[dashed] (5,0)--(10,0);
\draw[dashed] (8,0.6)--+(30:4);
\draw[red,thick,->] (8.7,0) arc[radius=1.7, start angle=0, end angle=30];
\draw[blue,fill] (6,0) circle (3pt);
\node at (6,-0.6) {$V_{ni}$};
\draw[blue,fill] (8,0.6) circle (3pt);
\node at (8,1.2) {$V_{fn}$};
\draw[dashed] (6,0)--(8,0.6);
\draw[very thick,->] (4,0)--+(0:1) node[above] {$i$};
\draw[very thick,->] (9,1.17)--+(30:1) node[above] {$f$};
\node at (7,0.6) {$n$};
\end{scope}

\begin{scope}[shift={(4,-5)}]
\draw[dashed] (5,0)--(10,0);
\draw[dashed] (8,0.6)--+(30:4);
\draw[red,thick,->] (8.7,0) arc[radius=1.7, start angle=0, end angle=30];
\draw[blue,fill] (6,0) circle (3pt);
\node at (6,-0.6) {$V_{mi}$};
\draw[blue,fill] (8,0.6) circle (3pt);
\node at (7,1.2) {$V_{nm}$};
\draw[blue,fill] (7,0.6) circle (3pt);
\node at (8,1.2) {$V_{fn}$};
\draw[dashed] (6,0)--(7,0.6) -- (8,0.6);
\draw[very thick,->] (4,0)--+(0:1) node[above] {$i$};
\draw[very thick,->] (9,1.17)--+(30:1) node[above] {$f$};
\node at (7.5,0.4) {$n$};
\node at (6.25,0.45) {$m$};
\end{scope}

\end{tikzpicture}
\caption{Graphical representation of the perturbation series in the $i\rightarrow f$ transition
to first order, second order, and third order.}
\end{center}
\end{figure}
%%%%%%%%%%%%%%%%   END FIGURE  %%%%%%%%%%%%%%%%%%%%%%%%%%%
%
\vskip 5cm
\doclicenseThis

\end{document}
