%%%%%%%%%%%%%%%%%%%%%%%%%%%%%%%%%%%%%%%%%%%%%%%%%%%
%% P3: Phenomenology of Particle Physics                         
%%
%% Author:  André Rubbia                   		 
%%
%% Figure 4.9 Quantum-mechanical description of the scattering experiment. 
%%
%% This work is licensed under the Creative Commons Attribution 4.0 International License. 
%% To view a copy of this license, visit http://creativecommons.org/licenses/by/4.0/ or 
%% send a letter to Creative Commons, PO Box 1866, Mountain View, CA 94042, USA.
%%
%%%%%%%%%%%%%%%%%%%%%%%%%%%%%%%%%%%%%%%%%%%%%%%%%%%

\documentclass[a4paper,10pt]{article}
\usepackage{hyperref}
\usepackage{hyperxmp}
\usepackage[
    type={CC},
    version={4.0}
]{doclicense}

\usepackage[T1]{fontenc}
\usepackage[utf8]{inputenc}
\usepackage{lmodern}
\usepackage[labelfont=bf]{caption}
\usepackage{upgreek}

\usepackage{tikz}
\usepackage{pgfplots}
\pgfplotsset{compat=1.17}
\usepgfplotslibrary{ternary}
\usepgfplotslibrary{fillbetween}
\usepgfplotslibrary{external}

\def\d{\mathrm{d}}

\begin{document}

%%%%%%%%%%%%%%%%   FIGURE  %%%%%%%%%%%%%%%%%%%%%%%%%%%%%%
\begin{figure}[htb]
\centering
\begin{tikzpicture}
\foreach \x in {0,...,5}
    \draw[blue,dotted,thick] (\x*0.5,-2)--(\x*0.5,2);
\draw[dashed] (0,0)--(10,0);
\draw[dashed] (7,0)--+(30:9);
\draw[red,thick,->] (8.7,0) arc[radius=1.7, start angle=0, end angle=30];
\foreach \x in {0,...,5}
    \draw[blue,dotted,thick] (\x*0.5+12,\x*0.5*0.5+1)--+(120:4);
%\draw[gray,fill] (7,0) circle (3pt);
\node at (7,-0.3) {$V$};
\node[red] at (9,0.6) {$\theta$};
\draw[blue,fill] (4,0) circle (3pt);
\draw[very thick,->] (4,0)--+(0:1) node[above] {$\vec p_i$};
\draw[blue,fill] (9,1.15) circle (3pt);
\draw[very thick,->] (9,1.15)--+(30:1) node[above] {$\vec p_f$};

\draw[blue,fill] (2,4) circle (3pt);
\draw[thick,->] (2,4)--+(0:1) node[below] {$\vec p_i$};
\draw[thick,->] (2,4)--+(30:1) node[above] {$\vec p_f$};
\draw[red,thick,->] (2.5,4) arc[radius=0.5, start angle=0, end angle=30];
\node[red] at (3,4.3) {$\theta$};
\node[red] at (2.5,3) {$\vec p_i\cdot \vec p_f = p^2 \cos\theta$};
\end{tikzpicture}
\caption{Quantum-mechanical description of the scattering experiment. The angle $\theta$ represents the scattering angle or the deflection and
we assumed an elastic scattering $|\vec{p_i}| = |\vec{p_f}| = p$.}
\end{figure}
%%%%%%%%%%%%%%%%   END FIGURE  %%%%%%%%%%%%%%%%%%%%%%%%%%%
%
\vskip 5cm
\doclicenseThis

\end{document}
