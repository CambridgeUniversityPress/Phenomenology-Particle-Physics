%%%%%%%%%%%%%%%%%%%%%%%%%%%%%%%%%%%%%%%%%%%%%%%%%%%
%% P3: Phenomenology of Particle Physics                         
%%
%% Author:  André Rubbia                   		 
%%
%% Figure 4.8 Experimental layout to test the Bohm--Aharonov effect.
%%
%% This work is licensed under the Creative Commons Attribution 4.0 International License. 
%% To view a copy of this license, visit http://creativecommons.org/licenses/by/4.0/ or 
%% send a letter to Creative Commons, PO Box 1866, Mountain View, CA 94042, USA.
%%
%%%%%%%%%%%%%%%%%%%%%%%%%%%%%%%%%%%%%%%%%%%%%%%%%%%

\documentclass[a4paper,10pt]{article}
\usepackage{hyperref}
\usepackage{hyperxmp}
\usepackage[
    type={CC},
    version={4.0}
]{doclicense}

\usepackage[T1]{fontenc}
\usepackage[utf8]{inputenc}
\usepackage{lmodern}
\usepackage[labelfont=bf]{caption}
\usepackage{upgreek}
\usepackage{amssymb}
\usepackage{amsmath}

\usepackage{tikz}
\usepackage{pgfplots}
\pgfplotsset{compat=1.17}
\usepgfplotslibrary{ternary}
\usepgfplotslibrary{fillbetween}
\usepgfplotslibrary{external}

\usepackage{pst-3dplot}
\usepackage{auto-pst-pdf}

\def\d{\mathrm{d}}

\begin{document}

%%%%%%%%%%%%%%%   FIGURE  %%%%%%%%%%%%%%%%%%%%%%%%%%%%%%
\begin{figure}[htb]
\centering
\psset{unit=0.75}
\begin{pspicture}(-4,-4)(4,4)%
\psset{Beta=40}
%\pstThreeDCoor[xMin=-4,xMax=5,yMin=0,yMax=5,zMin=0,zMax=4]
%%\psdot[dotscale=10](0,0)
%% \pstThreeDCircle[ O p t i o n s ] ( c x , c y, c z ) ( u x , u y, u z ) ( v x , v y, v z )
% back screen
\psset{linecolor=gray,linewidth=1pt,fillcolor=black!10,fillstyle=solid}
\pstThreeDSquare(-7,-2.5,0)(0,5,0)(0,0,5)
% trajectories
\psset{arrows=->,linewidth=3pt,fillstyle=none}
\pstThreeDLine[linecolor=magenta, linewidth=3pt](12,-2.5,2.5)(7,-2.5,2.5)(-7,-0.25,2.5)
% A-field
\psset{linecolor=blue,arrows=->,linewidth=2pt,fillstyle=none}
\pstThreeDCircle(0,0,0)(1,0,0)(0,1,0)
\pstThreeDCircle(0,0,0)(2,0,0)(0,1,0)
\pstThreeDCircle(0,0,0)(3,0,0)(0,1,0)
\pstThreeDCircle(0,0,3)(1,0,0)(0,1,0)
\pstThreeDCircle(0,0,3)(2,0,0)(0,1,0)
\pstThreeDCircle(0,0,3)(3,0,0)(0,1,0)
%% solenoid
%% \pstIIIDCylinder(x,y,z){radius}{height}
\psset{linecolor=black,linewidth=1pt}
\pstIIIDCylinder[fillcolor=gray, fillstyle=solid,linecolor=black!20,increment=0.4](0,0,-3){0.25}{8}
\pstThreeDLine(0,0,5.5)(0,0,7.5)
\pstThreeDPut(0,0,8){$\vec B_{inside}$}
\pstThreeDPut(0,4,0){\blue$\vec A$}
\pstThreeDPut(2.5,-1,7){$\vec B_{outside}=0$}
\pstThreeDPut(0,0,-3.5){Solenoid}
% trajectories
\pstThreeDLine[linecolor=magenta, linewidth=3pt](12,-2.5,2.5)(7,2.5,2.5)(-7,-0.25,2.5)
%% front field lines
\psset{linecolor=blue,linewidth=2pt}
\pstThreeDCircle[beginAngle=0,endAngle=180](0,0,3)(1,0,0)(0,1,0)
\pstThreeDCircle[beginAngle=0,endAngle=180](0,0,3)(2,0,0)(0,1,0)
\pstThreeDCircle[beginAngle=0,endAngle=180](0,0,3)(3,0,0)(0,1,0)
\pstThreeDCircle[beginAngle=0,endAngle=180](0,0,0)(1,0,0)(0,1,0)
\pstThreeDCircle[beginAngle=0,endAngle=180](0,0,0)(2,0,0)(0,1,0)
\pstThreeDCircle[beginAngle=0,endAngle=180](0,0,0)(3,0,0)(0,1,0)
% front screen
\psset{linecolor=gray,no arrow,linewidth=1pt,fillcolor=black!10,fillstyle=solid}
\pstThreeDSquare(7,-4,0)(0,8,0)(0,0,5)
% slits
\psset{linecolor=gray,linewidth=1pt,fillcolor=white,fillstyle=solid}
\pstThreeDSquare(7,-2.5,0)(0,0.3,0)(0,0,5)
\pstThreeDSquare(7,2.5,0)(0,0.3,0)(0,0,5)
% trajectories
\psset{arrows=->,linewidth=3pt,fillstyle=none}
\pstThreeDLine[linecolor=magenta, linewidth=3pt](12,-2.5,2.5)(7,-2.5,2.5)
\pstThreeDLine[linecolor=magenta, linewidth=3pt](12,-2.5,2.5)(7,2.5,2.5)
\pstThreeDPut(12.5,-2.5,2.5){\magenta$e^-$}
% labels
\pstPlanePut[plane=yz](-7,0,0.5){Screen}
\pstPlanePut[plane=yz](7,-2.,0.5){Slit 1}
\pstPlanePut[plane=yz](7,1,0.5){Slit 2}
\pstPlanePut[plane=yz](7,-2.,2.5){\textbf{A}}
\pstPlanePut[plane=yz](7,2,2.5){\textbf{B}}
\pstPlanePut[plane=yz](-7,-0.25,2.5){\textbf{P}}
\end{pspicture}
\caption{Experimental layout to test the Bohm--Aharonov effect: a two-slit experiment with a solenoid placed between
the slits.}
\end{figure}
%%%%%%%%%%%%%%%  END FIGURE  %%%%%%%%%%%%%%%%%%%%%%%%%%%%%%%
\vskip 5cm
\doclicenseThis

\end{document}
