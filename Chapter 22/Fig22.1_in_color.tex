%%%%%%%%%%%%%%%%%%%%%%%%%%%%%%%%%%%%%%%%%%%%%%%%%%%
%% P3: Phenomenology of Particle Physics                         
%%
%% Author:  André Rubbia                   		 
%%
%% Figure 22.1 Angle between momentum $\vec p$ of a particle and the orientation of its spin $\vec s$ before
%% and after parity transformation.
%%
%% This work is licensed under the Creative Commons Attribution 4.0 International License. 
%% To view a copy of this license, visit http://creativecommons.org/licenses/by/4.0/ or 
%% send a letter to Creative Commons, PO Box 1866, Mountain View, CA 94042, USA.
%%
%%%%%%%%%%%%%%%%%%%%%%%%%%%%%%%%%%%%%%%%%%%%%%%%%%%

\documentclass[a4paper,10pt]{article}
\usepackage{hyperref}
\usepackage{hyperxmp}
\usepackage[
    type={CC},
    version={4.0}
]{doclicense}

\usepackage[T1]{fontenc}
\usepackage[utf8]{inputenc}
\usepackage{lmodern}
\usepackage[labelfont=bf]{caption}
\usepackage{upgreek}

\usepackage{tikz}
\usetikzlibrary{patterns}
\usetikzlibrary{decorations.pathmorphing}
\usetikzlibrary{decorations.markings}
\usetikzlibrary{arrows}
\usetikzlibrary{svg.path}
\usetikzlibrary{shapes}
\usetikzlibrary{arrows.meta}
% define the arrow style
\tikzset{
    arrow/.style={
        decoration={
            markings,
            mark=at position .5 with {
                \arrow[#1, scale=1.5]{latex}
            }
        },
        postaction={decorate},
    }
}
\tikzset{
    arrow flipped/.style={
        decoration={
            markings,
            mark=at position .5 with {
                \arrow[#1, scale=1.5]{latex reversed}
            }
        },
        postaction={decorate},
    }
}
\usepackage{pgfplots}
\pgfplotsset{compat=1.17}
\usepgfplotslibrary{ternary}
\usepgfplotslibrary{fillbetween}
\usepgfplotslibrary{external}

\def\d{\mathrm{d}}
\setlength{\oddsidemargin}{-1.0cm}
\setlength{\evensidemargin}{-1.0cm}
\setlength{\textheight}{25cm}
\setlength{\textwidth}{18cm}

\begin{document}

%%%%%%%%%%%%%%%%   FIGURE  %%%%%%%%%%%%%%%%%%%%%%%%%%%%%%
\begin{figure}[htb]
\begin{center}
\begin{tikzpicture}[scale=2]
\draw[thick,->] (0,0)  -- (0,2) node [left] {$\vec p$};
\draw[dashed,thin] (0,0.75) -- +(60:1) node [right] {$\vec s$};
\draw[rotate around={-30:(0,1)}] (0,0.5) -- ++(0,0.5) -- ++(-0.125,0) -- ++(0.25,0.25) -- ++(0.25,-0.25) -- ++(-0.125,0) -- ++(0,-0.5) -- ++(-0.25,0);
\draw[black,fill=black] (0,0.75) circle (.05);
\draw [blue,very thick](0,0.75) +(90:0.7) arc (90:60:0.7);
\draw[blue] (0,0.75) +(75:0.85) node {$\theta$};

\draw[very thick,->] (1.5,1)  -- (3,1) node [right] {$P$};

\draw[thick,->] (4,2)  -- (4,0) node [left] {$\vec p$};
\draw[dashed,thin] (4,1.25) -- +(60:1) node [right] {$\vec s$};
\draw[rotate around={-30:(4,1.5)}] (4,1) -- ++(0,0.5) -- ++(-0.125,0) -- ++(0.25,0.25) -- ++(0.25,-0.25) -- ++(-0.125,0) -- ++(0,-0.5) -- ++(-0.25,0);
\draw[black,fill=black] (4,1.25) circle (.05);
\draw [blue,very thick](4,1.25) +(60:0.7) arc (60:-90:0.7);
\draw[blue] (4,1.25) +(-60:0.85) node {$\pi-\theta$};
\end{tikzpicture}
\end{center}
\caption{Angle between momentum $\vec p$ of a particle and the orientation of its spin $\vec s$ before
and after parity transformation.}
\end{figure}
%
%%%%%%%%%%%%%%%%   END FIGURE  %%%%%%%%%%%%%%%%%%%%%%%%%%%%%%
%
\vskip 5cm
\doclicenseThis

\end{document}
