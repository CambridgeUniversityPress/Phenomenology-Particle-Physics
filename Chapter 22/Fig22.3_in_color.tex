%%%%%%%%%%%%%%%%%%%%%%%%%%%%%%%%%%%%%%%%%%%%%%%%%%%
%% P3: Phenomenology of Particle Physics                         
%%
%% Author:  André Rubbia                   		 
%%
%% Figure 22.3 $\beta$ disintegration of $^{60}\mathrm{Co}\rightarrow ^{60}\mathrm{Ni}^*+e^-+\bar \nu_e$.
%%
%% This work is licensed under the Creative Commons Attribution 4.0 International License. 
%% To view a copy of this license, visit http://creativecommons.org/licenses/by/4.0/ or 
%% send a letter to Creative Commons, PO Box 1866, Mountain View, CA 94042, USA.
%%
%%%%%%%%%%%%%%%%%%%%%%%%%%%%%%%%%%%%%%%%%%%%%%%%%%%

\documentclass[a4paper,10pt]{article}

\usepackage[T1]{fontenc}
\usepackage[utf8]{inputenc}
\usepackage{lmodern}
\usepackage[labelfont=bf]{caption}
\usepackage{upgreek}

\usepackage{tikz}
\usetikzlibrary{patterns}
\usetikzlibrary{decorations.pathmorphing}
\usetikzlibrary{decorations.markings}
\usetikzlibrary{arrows}
\usetikzlibrary{svg.path}
\usetikzlibrary{shapes}
\usetikzlibrary{arrows.meta}
% define the arrow style
\tikzset{
    arrow/.style={
        decoration={
            markings,
            mark=at position .5 with {
                \arrow[#1, scale=1.5]{latex}
            }
        },
        postaction={decorate},
    }
}
\tikzset{
    arrow flipped/.style={
        decoration={
            markings,
            mark=at position .5 with {
                \arrow[#1, scale=1.5]{latex reversed}
            }
        },
        postaction={decorate},
    }
}
\usepackage{pgfplots}
\pgfplotsset{compat=1.17}
\usepgfplotslibrary{ternary}
\usepgfplotslibrary{fillbetween}
\usepgfplotslibrary{external}

\def\d{\mathrm{d}}
\setlength{\oddsidemargin}{-1.0cm}
\setlength{\evensidemargin}{-1.0cm}
\setlength{\textheight}{25cm}
\setlength{\textwidth}{18cm}

\begin{document}

%%%%%%%%%%%%%%%%   FIGURE  %%%%%%%%%%%%%%%%%%%%%%%%%%%%%%
\begin{figure}[htb]
\begin{center}
\begin{tikzpicture}
\draw[thick,->] (0,0)  -- (0,3) node [left] {$\vec B$};
\draw[very thick,->] (1,0.5)  node [below] {$^{60}\mathrm{Co}$} -- (1,2.5) node [above] {$J=5$};
\draw[->] (2,1.5)  -- (3,1.5);
\draw[red,very thick,->] (4,1)  node [below] {$^{60}\mathrm{Ni}^*$} -- (4,2) node [above] {$J=4$};
\draw[blue,very thick,->] (5.5,1.25)  node [below] {$e^-$} -- (5.5,1.75) node [above] {$+1/2$};
\draw[magenta,very thick,->] (7,1.25)  node [below] {$\bar \nu$} -- (7,1.75) node [above] {$+1/2$};
\draw[dashed] (9,0)  -- (9,3);
\draw[blue, very thick,->] (11,1.25)  -- +(0,0.5);
\draw[thick,->] (11.2,0.5) node [below] {$e^-$} -- +(0,2) node [above] {$\vec p_e$};
\draw[blue, very thick,->] (14,1.25)  -- +(0,0.5);
\draw[thick,->] (14.2,2.5) node [above] {$e^-$} -- +(0,-2) node [below] {$\vec p_e$};
\node at (14.5,-0.5) {\bf enhanced};
\node at (11.5,-0.5) {\bf suppressed};
\end{tikzpicture}
\end{center}
\caption{$\beta$ disintegration of $^{60}\mathrm{Co}\rightarrow ^{60}\mathrm{Ni}^*+e^-+\bar \nu_e$:
spin orientation (in thick lines) and flight direction (thinner lines on the right-hand side). The $\vec B$-field defines
the axis of quantization. The electron can be emitted in the direction or opposite to the
direction of the magnetic field.}
\end{figure}
%
%%%%%%%%%%%%%%%%   END FIGURE  %%%%%%%%%%%%%%%%%%%%%%%%%%%%%%
%
\end{document}
