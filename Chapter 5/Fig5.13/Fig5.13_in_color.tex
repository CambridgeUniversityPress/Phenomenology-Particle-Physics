%%%%%%%%%%%%%%%%%%%%%%%%%%%%%%%%%%%%%%%%%%%%%%%%%%%
%% P3: Phenomenology of Particle Physics                         
%%
%% Author:  André Rubbia                   		 
%%
%% Figure 5.13 A modern Dalitz plot with 1000~events distributed according to phase space.
%%
%% This work is licensed under the Creative Commons Attribution 4.0 International License. 
%% To view a copy of this license, visit http://creativecommons.org/licenses/by/4.0/ or 
%% send a letter to Creative Commons, PO Box 1866, Mountain View, CA 94042, USA.
%%
%%%%%%%%%%%%%%%%%%%%%%%%%%%%%%%%%%%%%%%%%%%%%%%%%%%

\documentclass[a4paper,10pt]{article}
\usepackage{hyperref}
\usepackage{hyperxmp}
\usepackage[
    type={CC},
    version={4.0}
]{doclicense}

\usepackage[T1]{fontenc}
\usepackage[utf8]{inputenc}
\usepackage{lmodern}
\usepackage[labelfont=bf]{caption}
\usepackage{upgreek}

\usepackage{tikz}
\usepackage{pgfplots}
\pgfplotsset{compat=1.17}
\usepgfplotslibrary{ternary}
\usepgfplotslibrary{fillbetween}
\usepgfplotslibrary{external}
\pgfkeys{/pgf/number format/.cd,1000 sep={}}

\def\d{\mathrm{d}}

\setlength{\textwidth}{168mm}
\setlength{\parindent}{6mm}
\setlength{\oddsidemargin}{-1.0cm}
\setlength{\evensidemargin}{-1.0cm}

\begin{document}

%%%%%%%%%%%%%%%%   FIGURE  %%%%%%%%%%%%%%%%%%%%%%%%%%%%%%
\begin{figure}[h]
\centering
\begin{tikzpicture}[scale=1]
\begin{axis}[    title=$M\rightarrow $ 3 massless,
	xlabel=$m^2_{12}$~(GeV$^2$),
	ylabel=$m^2_{23}$~(GeV$^2$)]
    \addplot+[only marks, blue, mark=o] table [x=XM212, y=XM223] {dalitz_3zero.dat};
    \end{axis}
\end{tikzpicture}
\begin{tikzpicture}[scale=1]
\begin{axis}[    title=$M\rightarrow $ 3 massive,
	ymin=0, ymax=11000,
	xlabel=$m^2_{12}$~(GeV$^2$),
	ylabel=$m^2_{23}$~(GeV$^2$)]
    \addplot+[only marks, blue, mark=o] table [x=XM212, y=XM223] {dalitz_1_10_30.dat};
    \draw[dashed,very thick] (axis cs:1000,1600) -- (axis cs:4000,1600);
    \node[left] at (axis cs:1000,1600) {\tiny $(m_2+m_3)^2$};
    \draw[dashed,very thick] (axis cs:121,4000) -- (axis cs:121,10500);
    \node at (axis cs:250,10500) {\tiny $(m_1+m_2)^2$};
    \draw[dashed,very thick] (axis cs:0,9801) -- (axis cs:1000,9801);
    \node[right] at (axis cs:1000,9801) {\tiny $(M-m_1)^2$};
    \draw[dashed,very thick] (axis cs:4900,2000) -- (axis cs:4900,6000);
    \node[above] at (axis cs:4700,6000) {\tiny $(M-m_3)^2$};
    \draw[red,dotted,very thick] (axis cs:3000,0) -- (axis cs:3000,9000);
    \draw[red, very thick, <-] (axis cs:3000,1600) -- (axis cs:3500,1000) node[right] {$(m_{23}^2)_{min}$};
    \draw[red, very thick, <-] (axis cs:3000,6500) -- (axis cs:3500,9000) node[right] {$(m_{23}^2)_{max}$};
    \end{axis}
\end{tikzpicture}
\caption{A modern Dalitz plot with 1000~events distributed according to phase space. The populated area corresponds to the kinematically
allowed regions.
For illustration, the mass of the parent particle $M$ is 100~GeV. (left) Massless daughter particles; (right) the
masses were chosen to be $m_1=1~$GeV, $m_2=10~$GeV, and $m_3=30$~GeV,
as in Figure 5.11.}
\end{figure}
%
%%%%%%%%%%%%%%%%   END FIGURE  %%%%%%%%%%%%%%%%%%%%%%%%%%%%%%
\vskip 5cm
\doclicenseThis
%
\end{document}
