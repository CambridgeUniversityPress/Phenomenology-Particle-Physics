%%%%%%%%%%%%%%%%%%%%%%%%%%%%%%%%%%%%%%%%%%%%%%%%%%%
%% P3: Phenomenology of Particle Physics                         
%%
%% Author:  André Rubbia                   		 
%%
%% Figure 5.11 Ternary (also Dalitz) plot of a three-body decay.
%%
%% This work is licensed under the Creative Commons Attribution 4.0 International License. 
%% To view a copy of this license, visit http://creativecommons.org/licenses/by/4.0/ or 
%% send a letter to Creative Commons, PO Box 1866, Mountain View, CA 94042, USA.
%%
%%%%%%%%%%%%%%%%%%%%%%%%%%%%%%%%%%%%%%%%%%%%%%%%%%%

\documentclass[a4paper,10pt]{article}

\usepackage[T1]{fontenc}
\usepackage[utf8]{inputenc}
\usepackage{lmodern}
\usepackage[labelfont=bf]{caption}
\usepackage{upgreek}

\usepackage{tikz}
\usepackage{pgfplots}
\pgfplotsset{compat=1.17}
\usepgfplotslibrary{ternary}
\usepgfplotslibrary{fillbetween}
\usepgfplotslibrary{external}

\def\d{\mathrm{d}}

\begin{document}

%%%%%%%%%%%%%%%%   FIGURE  %%%%%%%%%%%%%%%%%%%%%%%%%%%%%%
\begin{figure}[htb]
\begin{center}
\begin{tikzpicture}[scale=0.9]
\begin{ternaryaxis}[    title=$M\rightarrow $ 3 massless, xlabel=$T_1/Q$, ylabel=$T_2/Q$, zlabel=$T_3/Q$,
	xticklabel={\pgfmathprintnumber\tick\%},yticklabel={\pgfmathprintnumber\tick\%},
	zticklabel={\pgfmathprintnumber\tick\%}]
    \addplot3[only marks, blue, mark=o] table [x=T1, y=T2, z=T3] {dalitz_3zero.dat};
    \end{ternaryaxis}
\end{tikzpicture}
\hspace{0.5cm}
\begin{tikzpicture}[scale=0.9]
\begin{ternaryaxis}[    title=$M\rightarrow $ 3 massive, xlabel=$T_1/Q$, ylabel=$T_2/Q$, zlabel=$T_3/Q$,
	xticklabel={\pgfmathprintnumber\tick\%},yticklabel={\pgfmathprintnumber\tick\%},
	zticklabel={\pgfmathprintnumber\tick\%}]
   \addplot3[only marks, blue, mark=o] table [x=T1, y=T2, z=T3] {dalitz_1_10_30.dat};
    \end{ternaryaxis}
\end{tikzpicture}
\caption{Ternary (also Dalitz) plot of a three-body decay. The  circles represent 1000~events in the
kinematically allowed region. Phase space predicts a uniform population in these regions.
For illustration, the mass of the parent particle $M$ is 100~GeV. (left) Massless daughter
particles; (right) the
masses were chosen to be $m_1=1~$GeV, $m_2=10~$GeV, and $m_3=30$~GeV.}
\end{center}
\end{figure}
%
%%%%%%%%%%%%%%%%   END FIGURE  %%%%%%%%%%%%%%%%%%%%%%%%%%%%%%%

\end{document}
